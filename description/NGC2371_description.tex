\begin{block}{NGC2371/2372, Planetary Nebula in Gem}
    RA: 07:25:34.66, Dec: +29:29:26.3 \\ 
    Dimensions: 2.2 x 0.96 PA: 125 \\ 
    Magnitude (V): 11.2

    Central Star: \\ 
      \hspace{1em}UCAC2 42051185 \\ 
      \hspace{1em}V-mag: 14.85 

    NGC2371 also known as the "Peanut Nebula" or the "Double Bubble" is double
    lobed planetary nebula oriented NE-SW with the lobes separated by a dark
    lane. It has a faint outer shell possibly visible using an OIII filter. The
    central star is very faint and likely difficult to see.

    It is located about 1° 40' north of ι Gem. There are a couple 9th to 10th
    magnitude stars in the field of view but it is fairly isolated.
    
    Finder fov: 30 

    \url{https://en.wikipedia.org/wiki/NGC_2371-2} 
\end{block}
